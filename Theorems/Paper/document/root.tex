\documentclass[runningheads]{llncs}
\usepackage{times}
\usepackage{isabelle}
\usepackage{isabellesym}
\usepackage{amsmath}
\usepackage{amssymb}
\usepackage{mathpartir}
\usepackage{tikz}
\usepackage{pgf}
\usetikzlibrary{positioning}
\usepackage{pdfsetup}
%%\usepackage{stmaryrd}
\usepackage{url}
\usepackage{color}

\titlerunning{POSIX Lexing with Derivatives of Regular Expressions}

\urlstyle{rm}
\isabellestyle{it}
\renewcommand{\isastyleminor}{\it}% 
\renewcommand{\isastyle}{\normalsize\it}%


\def\dn{\,\stackrel{\mbox{\scriptsize def}}{=}\,}
\renewcommand{\isasymequiv}{$\dn$}
\renewcommand{\isasymemptyset}{$\varnothing$}
\renewcommand{\isacharunderscore}{\mbox{$\_\!\_$}}
\renewcommand{\isasymiota}{\makebox[0mm]{${}^{\prime}$}}

\def\Brz{Brzozowski}
\def\der{\backslash}
\newtheorem{falsehood}{Falsehood}
\newtheorem{conject}{Conjecture}

\begin{document}

\title{POSIX {L}exing with {D}erivatives of {R}egular {E}xpressions (Proof Pearl)}
\author{Fahad Ausaf\inst{1} \and Roy Dyckhoff\inst{2} \and Christian Urban\inst{3}}
\institute{King's College London\\
		\email{fahad.ausaf@icloud.com}
\and University of St Andrews\\
		\email{roy.dyckhoff@st-andrews.ac.uk}
\and King's College London\\
		\email{christian.urban@kcl.ac.uk}}
\maketitle

\begin{abstract}

Brzozowski introduced the notion of derivatives for regular
expressions. They can be used for a very simple regular expression
matching algorithm.  Sulzmann and Lu cleverly extended this algorithm
in order to deal with POSIX matching, which is the underlying
disambiguation strategy for regular expressions needed in lexers.
Sulzmann and Lu have made available on-line what they call a
``rigorous proof'' of the correctness of their algorithm w.r.t.\ their
specification; regrettably, it appears to us to have unfillable gaps.
In the first part of this paper we give our inductive definition of
what a POSIX value is and show $(i)$ that such a value is unique (for
given regular expression and string being matched) and $(ii)$ that
Sulzmann and Lu's algorithm always generates such a value (provided
that the regular expression matches the string).  We also prove the
correctness of an optimised version of the POSIX matching
algorithm. Our definitions and proof are much simpler than those by
Sulzmann and Lu and can be easily formalised in Isabelle/HOL. In the
second part we analyse the correctness argument by Sulzmann and Lu and
explain why the gaps in this argument cannot be filled easily.\smallskip

{\bf Keywords:} POSIX matching, Derivatives of Regular Expressions,
Isabelle/HOL
\end{abstract}

\input{session}

\end{document}

%%% Local Variables:
%%% mode: latex
%%% TeX-master: t
%%% End:
